\documentclass[a4paper,11pt]{article}

\usepackage[T1]{fontenc}
\usepackage[utf8]{inputenc}
\usepackage{lmodern}
\usepackage{eurosym}
\usepackage{lastpage}
\usepackage{xspace}
\usepackage[margin=15mm,includehead,includefoot]{geometry}
\usepackage{fancyhdr}
\usepackage{booktabs}
\usepackage{graphicx}
\usepackage{multirow}
\usepackage{array}
\usepackage[usenames,dvipsnames,table]{xcolor}
\usepackage{csquotes}
\usepackage{pgfgantt}
\usepackage[nohyperlinks,nolist]{acronym}
\usepackage[stable]{footmisc}

\usepackage[compact]{titlesec}
% HotFix from http://tex.stackexchange.com/a/300259/84485
% Version1 of titlesec is not compatible with the latest texlive. 
% Either the titlesec package must be updated, or the following HotFix used:
\usepackage{etoolbox}
\makeatletter
\patchcmd{\ttlh@hang}{\parindent\z@}{\parindent\z@\leavevmode}{}{}
\patchcmd{\ttlh@hang}{\noindent}{}{}{}
\makeatother

\usepackage[style=verbose-inote,backend=bibtex]{biblatex}
\usepackage{hyperref}
\usepackage{amsfonts,amsmath,amssymb}
\usepackage{soul} % for smarter (word-wrapping) underlining
\setul{1pt}{.4pt} % 1pt below contents
\usepackage{lineno}
\usepackage{enumitem}
\usepackage{pifont}

\usepackage{helvet}
\renewcommand{\familydefault}{\sfdefault}

\newcommand{\TODO}[1]{{\textcolor{red}{[\textbf{TODO:} #1]}}}
\titlespacing\section{0pt}{10pt plus 4pt minus 2pt}{8pt plus 2pt minus 2pt}
\titlespacing\subsection{0pt}{8pt plus 4pt minus 2pt}{8pt plus 2pt minus 2pt}
% \titlespacing\subsubsection{0pt}{6pt plus 4pt minus 2pt}{0pt plus 2pt minus 2pt}

\titleformat*{\section}{\large\bfseries}
\titleformat*{\subsection}{\normalsize\bfseries}
\titleformat*{\subsubsection}{\normalsize\bfseries}
\let\oldfootnotesize\footnotesize
\renewcommand{\footnotesize}{\fontsize{8bp}{1em}\selectfont}
\renewcommand{\cite}{\autocite} % citations in footnotes
\bibliography{bibliography}

\headheight=14pt

% Explicitly set footnote font size to match call (i.e., 8pt).
% Taken from http://tex.stackexchange.com/a/249422/84485
\makeatletter
\renewcommand\footnotesize{%
   \@setfontsize\footnotesize\@ixpt{8}%
   \abovedisplayskip 8\p@ \@plus2\p@ \@minus4\p@
   \abovedisplayshortskip \z@ \@plus\p@
   \belowdisplayshortskip 4\p@ \@plus2\p@ \@minus2\p@
   \def\@listi{\leftmargin\leftmargini
               \topsep 4\p@ \@plus2\p@ \@minus2\p@
               \parsep 2\p@ \@plus\p@ \@minus\p@
               \itemsep \parsep}%
   \belowdisplayskip \abovedisplayskip
}
\makeatother

\hypersetup{
    pdftitle={H2020-MSCA-IF-2020},    % title
    pdfauthor={},
    colorlinks=true,
    citecolor=black,
    linkcolor=black,
    urlcolor=blue
  }

\newcommand{\markStartPageLimit}{%
\begin{flushright}
\textcolor{red}{{\bf START PAGE COUNT \--- MAX 10 PAGES\---  \underline{DO NOT} ADD INTRODUCTORY PAGES BEFORE THIS}\\ \vspace{-.2cm}
\rule{\textwidth}{0.5mm}
}\vspace{-.5cm}
\end{flushright}
}
\newcommand{\markEndPageLimit}{%
\vfill
\begin{flushright}
\textcolor{red}{{\large {\bf STOP PAGE COUNT \--- MAX 10 PAGES~~~~~}}\\
\rule{\textwidth}{0.5mm}
}
\end{flushright}
}


% To correctly align fancy headers.
% Courtesy of: http://tex.stackexchange.com/a/88136/84485
\makeatletter
\newcommand{\resetHeadWidth}{\f@nch@setoffs}
\makeatother

\pagestyle{fancy}
\fancyhead{}
\fancyhead[C]{\color{gray}{\ac{PropAcronym}\xspace\xspace - EF-ST / EF-CAR / EF-RI / EF-SE / GF\\
(Delete as appropriate and include as header on each page)}}
\fancyfoot{}
\fancyfoot[C]{\color{gray}{Part B - Page \thepage~of \pageref*{sec:letters}}}
\resetHeadWidth

\renewcommand{\headrulewidth}{0pt}


% For CV
\definecolor{lightgray}{gray}{0.8}
\newcolumntype{L}{>{\raggedleft}p{0.14\textwidth}}
\newcolumntype{R}{p{0.8\textwidth}}
\newcommand\VRule{\color{lightgray}\vrule width 0.5pt}


% Spacing between paragraphs
\setlength{\parindent}{0em}
\setlength{\parskip}{4pt}


%List of acronyms used in the proposal 
%read the acronym package manual for details: http://ctan.mackichan.com/macros/latex/contrib/acronym/acronym.pdf
%\acs{PropAcronym} gives acronym [in square brackets]
%\acl{PropAcronym} gives long form of acronym, probably not used {in second set of braces}
%\acf{PropAcronym} gives both
\begin{acronym}
\acro{IF}{Individual Fellowships}
\acro{PropAcronym}[PROPOSAL ACRONYM]{This is my proposal's acronym}
\end{acronym}

\begin{document}
%mark proposal acronym as used, such that the short-hand form is always used by default.
\acused{PropAcronym}
%\begin{linenumbers}

% !TEX root = ./IF-2020-Part_B.tex

\markStartPageLimit
\section{Excellence\footnote{Literature should be listed in footnotes, minimum font size 8. All literature references will count towards the page limit.}}
\label{sec:excellence}


\subsection{Quality and credibility of the research/innovation project; level of novelty, appropriate consideration of inter/multidisciplinary and gender aspects}
\label{sec:excellence_quality}

Provide an introduction, discuss the state-of-the-art, specific objectives and give an
overview of the action.

\medskip\noindent
Discuss the research methodology and approach, highlighting the type of research /
innovation activities proposed.

\medskip\noindent
Explain the originality and innovative aspects of the planned research as well as the
contribution that the action is expected to make to advancements within the research
field. Describe any novel concepts, approaches or methods that will be implemented.

\medskip\noindent
Discuss the interdisciplinary aspects of the action (if relevant).

\medskip\noindent
Discuss the gender dimension in the research content (if relevant). In research activities
where human beings are involved as subjects or end-users, gender differences may
exist. In these cases the gender dimension in the research content has to be addressed
as an integral part of the proposal to ensure the highest level of scientific quality.


\subsection{Quality and appropriateness of the training and of the two way transfer of knowledge between the researcher and the host}
\label{sec:excellence_transfer}

Outline how a two way transfer of knowledge will occur between the researcher and the host institution(s):
\begin{itemize}
\item Explain how the experienced researcher will gain new knowledge during the fellowship at the hosting organisation(s).
\item Outline the previously acquired knowledge and skills that the researcher will transfer to the host organisation(s).
\end{itemize}

\noindent
For \textbf{Global Fellowships} explain how the newly acquired skills and knowledge in the Third Country will be transferred back to the host institution in Europe (the beneficiary) during the incoming phase.

\medskip\noindent
Describe the training that will be offered. Typical \textbf{training activities} in Individual Fellowships may include:
\begin{itemize}
\item Primarily, training-through-research by the means of an \ul{individual personalised
project}, under the guidance of the supervisor and other members of the research
staff of the host organisation(s)
\item Hands-on training activities for developing scientific skills (new techniques,
instruments, research integrity, 'big data'/'open science') and transferrable skills
(entrepreneurship, proposal preparation, patent applications,
management of IPR, project management, task coordination, supervising and
monitoring, take up and exploitation of research results)
\item Inter-sectoral or interdisciplinary transfer of knowledge (e.g. through secondments)
\item Participation in the research and financial management of the action
\item Organisation of scientific/training/dissemination events
\item Communication, outreach activities and horizontal skills
\item Training dedicated to gender issues
\end{itemize}

\medskip\noindent
\setlength{\fboxsep}{1mm}
\fbox{\parbox{\textwidth}{
A \textbf{Career Development Plan} should not be included in the proposal, but will be
part of the action's implementation in line with the European Charter for
Researchers. It should aim at achieving a realistic and well-defined objective in
terms of career advancement (e.g. attaining a leading independent position) or
resuming a research career after a break. The plan should be devised with the final
outcome to develop and significantly widen the competences of the experienced
researcher, particularly in terms of multi/interdisciplinary expertise, inter-sectoral
experience and transferable skills.
}}


\subsection{Quality of the supervision and of the integration in the team/institution}
\label{sec:excellence_supervision}

Describe the qualifications and experience of the supervisor(s). Provide information
regarding the supervisors' level of experience on the research topic proposed and their
track record of work, including main international collaborations, as well as the level
of experience in supervising/training especially at advanced level (PhD, postdoctoral
researchers). Information provided should include participation in projects,
publications, patents and any other relevant results.

\medskip\noindent
Describe the hosting arrangements\footnote{The hosting arrangements refer to the integration
of the researcher to his new environment in the premises of the host. 
It does not refer to the infrastructure of the host as described in
the Quality and efficiency of the implementation criterion.}.
The application must show that the experienced
researcher will be well-integrated within the team/institution so that all parties gain
maximum knowledge and skills from the fellowship. The nature and the quality of the
research group/environment as a whole should be outlined, together with the measures
taken to integrate the researcher in the different areas of expertise, disciplines, and
international networking opportunities that the host could offer.

\medskip\noindent
For \textbf{Global Fellowships} both phases should be described - for the outgoing phase,
specify the practical arrangements in place to host a researcher coming from another
country, and for the incoming phase specify the measures planned for the successful
(re)integration of the researcher.


\subsection{Potential of the researcher to reach or re-enforce professional maturity/independence during the fellowship}
\label{sec:excellence_maturity}

Researchers should \textbf{demonstrate} how their existing professional experience, talents
and the proposed research will contribute to their development as independent/mature
researchers, \ul{\textbf{during the fellowship}}. Explain the new competences and skills that will
be acquired and how they relate to the researcher’s existing professional experience.

\medskip\noindent
Please keep in mind that the fellowships will be awarded to the most talented
researchers as shown by the proposed research and their track record (Curriculum
Vitae, section~\ref{sec:cv}), in relation to their level of experience.




\section{Impact}
\label{sec:impact}

\subsection{ Enhancing the future career prospects of the researcher after the fellowship}
\label{sec:impact_researcher}

Explain the expected impact of the planned research and training (i.e. the added value
of the fellowship) on the future career prospects of the experienced researcher \ul{\textbf{after
the fellowship}}. Focus on how the new competences and skills (as explained in \ref{sec:excellence_maturity})
can make the researcher more successful in their long-term career.


\subsection{Quality of the proposed measures to exploit and disseminate the project results}
\label{sec:impact_dissemination}

Describe how the new knowledge generated by the action will be disseminated and
exploited, and what the potential impact is expected to be. Discuss the strategy for
targeting peers (scientific, industry and other actors, professional organisations, policy
makers, etc.) and to the wider community. Also describe potential commercialisation,
if applicable, and how intellectual property rights will be dealt with, where relevant.

\medskip\noindent
For more details refer to the
\href{http://ec.europa.eu/research/participants/docs/h2020-funding-guide/grants/grant-management/dissemination-of-results_en.htm}
{"Dissemination \& exploitation" section of the H2020 Online Manual}.

\medskip\noindent
Concrete planning for exploitation and dissemination activities must be included in the
Gantt chart.


\subsection{Quality of the proposed measures to communicate the project activities to different target audiences}
\label{sec:impact_communication}

Demonstrate how the planned public engagement activities contribute to creating
awareness of the performed research. Demonstrate how both the research and results
will be made known to the public in such a way they can be understood by non-specialists.

\medskip\noindent
The type of outreach activities could range from an Internet presence, press articles
and participating in European Researchers' Night events to presenting science,
research and innovation activities to students from primary and secondary schools or
universities in order to develop their interest in research careers.

\medskip\noindent
For more details, see the guide on \href{http://ec.europa.eu/research/participants/data/ref/h2020/other/gm/h2020-guide-comm_en.pdf}
{Communicating EU research and innovation guidance for project participants}
as well as the
\href{http://ec.europa.eu/research/participants/docs/h2020-funding-guide/grants/grant-management/communication_en.htm}
{"communication" section of the H2020 Online Manual}.

\medskip\noindent
Concrete planning for communication activities must be included in the Gantt chart.




\section{Quality and Efficiency of the Implementation}
\label{sec:implementation}

\subsection{Coherence and effectiveness of the work plan, including appropriateness of the allocation of tasks and resources}
\label{sec:implementation_work_plan}

Describe how the work planning and the resources mobilised will ensure that the
research and training objectives will be reached. Explain why the number of person-
months planned and requested for the project is appropriate in relation to the proposed
activities.

\medskip\noindent
Additionally, a Gantt chart must be included in the text listing the following:

\begin{itemize}
  \item Work Packages titles (there should be at least 1 WP);
  \item Indication of major deliverables, if applicable;
  \footnote{A \textbf{deliverable} is a distinct output of the action, meaningful in terms of the action’s overall objectives and may be a report, a document, a
  technical diagram, a software, etc. Deliverable numbers should be ordered according to delivery dates. Use the numbering convention <WP
  number>.<number of deliverable within that WP>. For example, deliverable 4.2 would be the second deliverable from work package 4.}
  \item Indication of major milestones, if applicable;
  \footnote{\textbf{Milestones} are control points in the action that help to chart progress. Milestones may correspond to the completion of a key deliverable,
  allowing the next phase of the work to begin. They may also be needed at intermediary points so that, if problems have arisen, corrective
  measures can be taken. A milestone may be a critical decision point in the action where, for example, the researcher must decide which of several
  technologies to adopt for further development.}
  \item Secondments, if applicable.
\end{itemize}

\medskip\noindent
The schedule should be in terms of number of months elapsed from the start of the
action.


\begin{figure}[!htbp]
\begin{center}

\begin{minipage}{0.9\textwidth}
\vspace{2pt}
\textbf{\footnotesize 
Notes:
\begin{itemize}
  \item The titles of the WP's indicated here do not have to be stricly followed or included in the Gantt chart for your specific proposal. Adapt as needed.
  \item The number of WPs provided here is an example only. Add or remove WP's as needed.
  \item Remove any columns for a duration longer than that of your proposal.
  \item Add as much detail as needed for your proposal.
\end{itemize}
}
\end{minipage}

\begin{ganttchart}[
    canvas/.append style={fill=none, draw=black!5, line width=.75pt},
    hgrid style/.style={draw=black!5, line width=.75pt},
    vgrid={*1{draw=black!5, line width=.75pt}},
    title/.style={draw=none, fill=none},
    title label font=\bfseries\footnotesize,
    title label node/.append style={below=7pt},
    include title in canvas=false,
    bar label font=\small\color{black!70},
    bar label node/.append style={left=2cm},
    bar/.append style={draw=none, fill=black!63},
    bar progress label font=\footnotesize\color{black!70},
    group left shift=0,
    group right shift=0,
    group height=.5,
    group peaks tip position=0,
    group label node/.append style={left=.6cm},
    group progress label font=\bfseries\small
  ]{1}{24}
  \gantttitle[
    title label node/.append style={below left=7pt and -3pt}
  ]{Month:\quad1}{1}
  \gantttitlelist{2,...,24}{1} \\
  \ganttgroup{Work Package}{1}{24} \\
  \ganttgroup{Deliverable}{24}{24} \\
  \ganttgroup{Milestone}{5}{5} \\
  \ganttgroup{Secondment}{20}{23} \\
  \ganttgroup{Short stay}{16}{16} \\
  \ganttgroup{Training}{5}{5} \\
  \ganttgroup{Dissemination}{23}{24} \\
  \ganttgroup{Communication}{12}{12} \\
  \ganttgroup{Other}{18}{21}
\end{ganttchart}

\end{center}
\caption{Example Gantt Chart}
\end{figure}


\subsection{Appropriateness of the management structure and procedures, including risk management}
\label{sec:implementation_management}

Describe the organisation and management structure, as well as the progress
monitoring mechanisms put in place, to ensure that objectives are reached. Discuss the
research and/or administrative risks that might endanger reaching the action objectives
and the contingency plans to be put in place should risk occur.

\medskip\noindent
If applicable, discuss any involvement of an entity with a capital or legal link to the
beneficiary (in particular, the name of the entity, type of link with the beneficiary and
tasks to be carried out).

\medskip\noindent
If needed, please indicate here information on the support services provided by the
host institution (European offices, HR services...).


\subsection{Appropriateness of the institutional environment (infrastructure)}
\label{sec:implementation_infrastructure}

The active contribution of the beneficiary to the research and training activities should
be described. For Global Fellowships the role of partner organisations in Third
Countries for the outgoing phase should also appear.

\medskip\noindent
Give a description of the main tasks and commitments of the beneficiary and all
partner organisations (if applicable).

\medskip\noindent
Describe the infrastructure, logistics, facilities offered insofar as they are necessary for
the good implementation of the action.


\markEndPageLimit

% !TEX root = ./IF-2020-Part_B.tex

\newpage
\section{CV of the experienced researcher}
\label{sec:cv}

The CV is intrinsic to the evaluation of the whole proposal and is assessed throughout the
three evaluation criteria by the expert evaluators. Ensure that the information provided in
Parts A and B is fully consistent. Always mention full dates (dd/mm/yyyy) in your CV.

\medskip\noindent
The CV should be limited to a maximum of 5 pages and should include \textbf{the standard
academic and research record}. Any research career gaps and/or unconventional paths
should be clearly explained so that this can be fairly assessed by the independent
evaluators. At a minimum, the CV should contain:

\begin{enumerate}[label=\alph*)]
\item the \textbf{name} of the researcher
\item \textbf{professional experience} (in \textbf{reverse} chronological order, using \textbf{exact} dates)
\item \textbf{education} (in reverse chronological order, using \textbf{exact} dates)
\end{enumerate}

\medskip\noindent
The CV should also include information on:

\begin{enumerate}
\item \textbf{Publications} in peer-reviewed scientific journals, peer-reviewed conference proceedings and/or monographs of their respective research fields, indicating also the number of citations (excluding self-citations) they have attracted.
\item Granted \textbf{patent(s)}.
\item \textbf{Research monographs, chapters} in collective volumes and any translations thereof.
\item \textbf{Invited presentations} to peer-reviewed, internationally established conferences and/or international advanced schools.
\item \textbf{Research expeditions} led by that the experienced researcher. 
\item \textbf{Organisation of International conferences} in the field of the researcher (membership in the steering and/or programme committee).
\item Examples of \textbf{participation in industrial innovation}.
\item \textbf{Prizes and Awards}.
\item \textbf{Funding} received so far.
\item \textbf{Supervising} and \textbf{mentoring} activities.
\end{enumerate}




\newpage
\section{Capacity of the Participating Organisations}
\label{sec:capacity}

\ul{List of participating organisations (one page)}

\medskip\noindent
Please provide a list of all participating organisations (the beneficiary and, where
applicable, the entity with a capital or legal link to the beneficiary and the partner
organisation
\footnote{All partner organisations should be listed here, including secondments}
) indicating the legal entity name, the department carrying out the work and
the supervisor.

\medskip\noindent
If a secondment in Europe is planned but the partner organisation is not yet known, as a
minimum the type of organisation planned (academic/non-academic) must be stated.

\medskip\noindent
Any inter-relationship between the participating organisation(s) or individuals and other
entities/persons (e.g. family ties, shared premises or facilities, joint ownership, financial
interest, overlapping staff or directors, etc.) \textbf{must} be declared and justified \textbf{in this part of
the proposal}.

\newcommand\MyHeadLeft[2]{\multicolumn{1}{|l|}{\parbox{#1}{\centering #2}}}
\newcommand\MyHead[2]{\multicolumn{1}{l|}{\parbox{#1}{\centering #2}}}

\medskip
\medskip
\noindent\begin{tabular}{|l|m{1cm}|c|c|c|}
\hline
  \MyHeadLeft{2.4cm}{\textbf{Participating\\organisations}}
& \MyHead{1cm}{\textbf{Legal\\Entity\\Short\\Name}}
& \textbf{Country}
& \textbf{Supervisor}
& \textbf{Role of partner organisation\footnotemark} \\
\hline
\ul{Beneficiary} & & & & \\\hline
- NAME  & & & & \\\hline
\ul{Entity with a capital or legal link} & & & & \\\hline
- NAME  & & & & \\\hline
\ul{Partner Organisation} & & & & \\\hline
- NAME  & & & & \\\hline
\end{tabular}
\vspace{\baselineskip}
\footnotetext{For example hosting secondments, for GF hosting the outgoing phase, etc.}


\begin{table}[h!]
{\fontsize{9bp}{1em}\selectfont % should be 9pt
\noindent\begin{tabular}{|>{\raggedright}p{.25\textwidth}|p{.7\textwidth}|}\hline
\multicolumn{2}{|l|}{\cellcolor{gray!50}
\begin{minipage}{0.90\textwidth}
\vspace{2pt}
1 page for each role \---- chose one of:
\begin{itemize}[noitemsep,topsep=3pt]
\item beneficiary (compulsory)
\item entity with a capital or legal link to the beneficiary (optional)
\item partner organisation for GF (compulsory for GF only)
\item partner organisation for secondment (optional)
\end{itemize}
\vspace{2pt}
\end{minipage}}
\\\hline
\multicolumn{2}{|c|}{\cellcolor{gray!50}\textbf{[Full name + Legal Entity Short Name + Country]}} \\\hline
\textbf{General Description} &

\\\hline
\textbf{Academic organisation} &
(Yes / No) delete as appropriate
\\\hline
\textbf{Role and profile of key persons (supervisor)} &
{\em (names, title, qualifications of the main supervisor)}
\\\hline
\textbf{Dept./Division / Laboratory} &

\\\hline
\textbf{Key research facilities, Infrastructure and Equipment} &
{\em Demonstrate that the beneficiary has sufficient facilities and
infrastructure to host and/or offer a suitable environment for
training and transfer of knowledge to the recruited
experienced researcher.
\newline
If applicable, indicate the name of the entity with a capital or
legal link to the beneficiary and its role in the action in the
following table.}
\\\hline
\textbf{Independent research premises?} &
{\em Explain the status of the beneficiary's research facilities \---- i.e.
are they owned by the beneficiary or rented by it? Are its
research premises wholly independent from other entities?
\newline
If applicable, indicate the name of the entity with a capital or
legal link to the beneficiary and describe the nature of the
link in the following table.}
\\\hline
\textbf{Previous and current involvement in research and training programmes} &
{\em Indicate up to 5 \textbf{relevant} EU, national or international
research and training actions/projects in which the
beneficiary has previously participated and/or is currently
participating.}
\\\hline
\textbf{Relevant publications and/or research/innovation products} &
{\em (Max 5) Only list items (co-)produced by the supervisor}
\\\hline
\end{tabular}}
\end{table}




\newpage
\section{Ethical Issues}
\label{sec:ethics}

Compliance with the relevant ethics provisions is essential from the beginning to the end of
the action and is an integral part of research funded by the European Union within Horizon 2020. 

\medskip\noindent
Applicants submitting research proposals for funding with Marie Sk\l{}odowska-Curie actions in
Horizon 2020 should demonstrate proactively that they are aware of, and will comply with, European
and national legislation and fundamental ethical principles, including those reflected in the 
\href{http://www.europarl.europa.eu/charter/pdf/text_en.pdf}{Charter of Fundamental Rights of the European Union}
and the \href{http://www.echr.coe.int/Documents/Convention_ENG.pdf}{European Convention on Human Rights and its Supplementary Protocols}.

\medskip\noindent
Please be aware that it is the applicants' responsibility to identify any potential ethical issue, 
to handle the ethical aspects of the proposal and to detail how these aspects will be addressed.

\bigskip\noindent
{\large {\bf \ul{The Ethics Review Procedure in Horizon 2020}}}

\medskip\noindent
All proposals above threshold and considered for funding will be subject to an Ethics Review carried out by independent ethics experts. 
When submitting a proposal to Horizon 2020, all applicants are required to complete an Ethics Issues Table (EIT) in the Part A of the proposal. 
Applicants who flag ethical issues in the EIT have to also complete a more in-depth Ethics Self-Assessment in Part B.

\medskip\noindent
The ethics self-assessment will become part of the Grant Agreement and may thus lead to
binding obligations. The Grant Agreement can only be signed if all ethics requirement
have been duly addressed. The ethics review result will distinguish between ethics
requirements to be addressed before Grant Agreement signature and those that can be
cleared at a later stage (e.g. ethics approvals to be submitted before the start of the action
task). In the latter case, a separate work package 'Ethics Requirements' listing the
deliverables will be created automatically.

\medskip\noindent
\setlength{\fboxsep}{1mm}
\fbox{\parbox{\textwidth}{
For more details, please refer to the H2020
\href{http://ec.europa.eu/research/participants/data/ref/h2020/grants_manual/hi/ethics/h2020_hi_ethics-self-assess_en.pdf}{``How to complete your Ethics Self-Assessment''} guide.
}}

\bigskip\noindent
{\large {\bf \ul{Ethics Self-Assessment (Part B)}}}

\medskip\noindent
The Ethics Self-Assessment must:

{\bf
\begin{enumerate}[leftmargin=*, label=\arabic*)]
  \item Describe how the proposal meets the EU and national legal and ethics requirements of the country/countries where the task raising ethical is to be carried out.
\end{enumerate}
}

\medskip\noindent
For more information on how to deal with Third Countries\footnote{In the context of ethics
appraisal, Third Country refers to non-EU country; Associated Countries are "ethics"
TC} please see Article 34 of the
\href{http://ec.europa.eu/research/participants/data/ref/h2020/grants_manual/amga/h2020-amga_en.pdf}{Annotated Model Grant Agreement},
as well as the
\href{http://ec.europa.eu/justice/data-protection/international-transfers/adequacy/index_en.htm}{rules for the protection of personal data inside and outside the EU}.
Please ensure and confirm that the research performed outside the EU
is compatible with the Union, National and International legislation and could have
been legally conducted in one of the EU Member States.

\medskip\noindent
Please list the documents provided with their expiry date.

\medskip\noindent
Ensure early compliance of the proposed research with EU and national legislation on
ethics in research. Should your proposal be selected for funding, you will be required - if
applicable - to confirm that you have obtained the following documents needed for
implementing the action tasks in question:

\begin{enumerate}[label=(\alph*)]
  \item any ethics committee opinion required under national law and
  \item any notification or authorisation for activities raising ethical issues required under national and/or European law
\end{enumerate}

\medskip\noindent
under national and/or European law
If you have not already applied for/received the ethics approval/required ethics
documents when submitting the proposal, please indicate in this section the approximate
date when you will obtain the relevant approvals/authorisations and any other ethics
documents. Please state explicitly that you will not proceed with any research with ethical
implications before obtaining the necessary authorizations/opinions.

\medskip\noindent
\fbox{\parbox{\textwidth}
{\em The documents must be kept on file and be submitted upon request by the beneficiary to
the REA (see Article 52). If they are not in English, they must be submitted together with
an English summary, which shows that the action tasks in question are covered and
includes the conclusions of the committee or authority concerned (if available).

\medskip\noindent
If you plan to request these ethics documents specifically for your proposed action, your
request must contain an explicit reference to the action's title.
}}

\bigskip
{\bf 
\begin{enumerate}[leftmargin=*, label=\arabic*), start=2]
  \item Explain in detail how you intend to address the ethical issues flagged, in particular with regard to: 
  \begin{itemize}
    \item {\normalfont the research {\bf objectives} (e.g., study of vulnerable 
    populations, cooperation with a Third Country, etc.);}
    \item {\normalfont the research {\bf methodology} (e.g., clinical trials, 
    involvement of children and related information and consent/assent procedures, data protection and privacy issues related to data collected, etc.);}
    \item {\normalfont the potential {\bf impact} of the research (e.g. dual use issues,
    environmental damage, malevolent use, etc.);}
    \item {\normalfont appropriate health and safety procedures - conforming to relevant
    local/national guidelines/legislation - for the staff involved;}
    \item {\normalfont  possible harm to the environment the research might cause
    (e.g. environmental risks of nanomaterials), and measures that will be taken to mitigate the risks.}
  \end{itemize}
\end{enumerate}
}

\noindent
In order to facilitate the ethics review of the proposal, you may wish to include in this section
one of the following statements (if relevant/applicable). Please delete as appropriate:

\begin{table}[h!]
{\fontsize{9bp}{1em}\selectfont % minimum size of 8pt
\noindent\begin{tabular}{|>{\raggedright}p{.85\textwidth}|cc|}\hline
\multicolumn{3}{|l|}{\cellcolor{gray!50} \textbf{Humans}} \\\hline
I confirm that templates of the informed consent forms and information
sheets (in language and terms intelligible to the participants) will be kept on file.
& Yes & No \\\hline
\multicolumn{3}{|l|}{\cellcolor{gray!50} \textbf{Animals}} \\\hline
I confirm that training certificates/personal licenses of the staff involved in
animal experiments have been obtained and will be kept on file.
& Yes & No \\\hline
I confirm that relevant authorisations for animal experiments (covering also
the work with genetically modified animals, if applicable) have been
obtained, and will be kept on file.
& Yes & No \\\hline
\multicolumn{3}{|l|}{\cellcolor{gray!50} \textbf{Environmental protection and safety}} \\\hline
I confirm that appropriate health and safety procedures conforming to
relevant local/national guidelines/legislation are followed for staff involved
in this project.
& Yes & No \\\hline
I confirm that authorisations for relevant facilities (e.g. security
classification of laboratory, GMO authorisation) have been obtained, and
will be kept on file.
& Yes & No \\\hline
\multicolumn{3}{|l|}{\cellcolor{gray!50} \textbf{Third country}} \\\hline
I confirm that the research performed outside the EU is compatible with
the Union, National and International legislation and could have been
legally conducted in one of the EU Member States.
& Yes & No \\\hline
\multicolumn{3}{|l|}{\cellcolor{gray!50} \textbf{Data protection}} \\\hline
I confirm that a Data Protection Officer (DPO) has been appointed and the
contact details of the DPO are made available to all data subjects involved
in the research.
& Yes & No \\\hline
I confirm that data intended to be processed is relevant and limited to the
purposes of the research project (in accordance with the 'data minimisation'
principle).
& Yes & No \\\hline
I confirm that relevant authorisations for further processing of previously
collected personal data have been obtained and will be kept on file.
& Yes & No \\\hline
I confirm that the data used are publicly available.
& Yes & No \\\hline
\end{tabular}}
\end{table}




\newpage
\section{Letter of Commitment (GF only)}
\label{sec:letters}

For Global Fellowship proposals, a {\em letter of commitment} \textbf{of the partner
organisations} (hosting the outgoing phase in a Third country) must be included in Part
B-2 to ensure their real and active participation. 
Do not attach this letter as a separate PDF file or as an
embedded file since this makes them invisible in the proposal. GF Proposals which fail to
include a letter of commitment of the partner organisation will be declared \textbf{inadmissible}.

\medskip\noindent
Minimum requirements for the letter of commitment:

\begin{itemize}
  \item heading or stamp from the institution;
  \item up-to-date (may not be dated prior to the call publication);
  \item the text must demonstrate the will to actively participate in the (identified) proposed action and the precise role.
\end{itemize}

\noindent
Please note that no template for this letter is provided, only general indications.



%\end{linenumbers}
\end{document}


