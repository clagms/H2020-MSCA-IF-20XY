% !TEX root = ./IF-2020-Part_B.tex

\markStartPageLimit
\section{Excellence\footnote{Literature should be listed in footnotes, minimum font size 8. All literature references will count towards the page limit.}}
\label{sec:excellence}


\subsection{Quality and credibility of the research/innovation project; level of novelty, appropriate consideration of inter/multidisciplinary and gender aspects}
\label{sec:excellence_quality}

Provide an introduction, discuss the state-of-the-art, specific objectives and give an
overview of the action.

Example citation \cite{Gomes2018}.

\medskip\noindent
Discuss the research methodology and approach, highlighting the type of research /
innovation activities proposed.

\medskip\noindent
Explain the originality and innovative aspects of the planned research as well as the
contribution that the action is expected to make to advancements within the research
field. Describe any novel concepts, approaches or methods that will be implemented.

\medskip\noindent
Discuss the interdisciplinary aspects of the action (if relevant).

\medskip\noindent
Discuss the gender dimension in the research content (if relevant). In research activities
where human beings are involved as subjects or end-users, gender differences may
exist. In these cases the gender dimension in the research content has to be addressed
as an integral part of the proposal to ensure the highest level of scientific quality.


\subsection{Quality and appropriateness of the training and of the two way transfer of knowledge between the researcher and the host}
\label{sec:excellence_transfer}

Outline how a two way transfer of knowledge will occur between the researcher and the host institution(s):
\begin{itemize}
\item Explain how the experienced researcher will gain new knowledge during the fellowship at the hosting organisation(s).
\item Outline the previously acquired knowledge and skills that the researcher will transfer to the host organisation(s).
\end{itemize}

\noindent
For \textbf{Global Fellowships} explain how the newly acquired skills and knowledge in the Third Country will be transferred back to the host institution in Europe (the beneficiary) during the incoming phase.

\medskip\noindent
Describe the training that will be offered. Typical \textbf{training activities} in Individual Fellowships may include:
\begin{itemize}
\item Primarily, training-through-research by the means of an \ul{individual personalised
project}, under the guidance of the supervisor and other members of the research
staff of the host organisation(s)
\item Hands-on training activities for developing scientific skills (new techniques,
instruments, research integrity, 'big data'/'open science') and transferrable skills
(entrepreneurship, proposal preparation, patent applications,
management of IPR, project management, task coordination, supervising and
monitoring, take up and exploitation of research results)
\item Inter-sectoral or interdisciplinary transfer of knowledge (e.g. through secondments)
\item Participation in the research and financial management of the action
\item Organisation of scientific/training/dissemination events
\item Communication, outreach activities and horizontal skills
\item Training dedicated to gender issues
\end{itemize}

\medskip\noindent
\setlength{\fboxsep}{1mm}
\fbox{\parbox{\textwidth}{
A \textbf{Career Development Plan} should not be included in the proposal, but will be
part of the action's implementation in line with the European Charter for
Researchers. It should aim at achieving a realistic and well-defined objective in
terms of career advancement (e.g. attaining a leading independent position) or
resuming a research career after a break. The plan should be devised with the final
outcome to develop and significantly widen the competences of the experienced
researcher, particularly in terms of multi/interdisciplinary expertise, inter-sectoral
experience and transferable skills.
}}


\subsection{Quality of the supervision and of the integration in the team/institution}
\label{sec:excellence_supervision}

Describe the qualifications and experience of the supervisor(s). Provide information
regarding the supervisors' level of experience on the research topic proposed and their
track record of work, including main international collaborations, as well as the level
of experience in supervising/training especially at advanced level (PhD, postdoctoral
researchers). Information provided should include participation in projects,
publications, patents and any other relevant results.

\medskip\noindent
Describe the hosting arrangements\footnote{The hosting arrangements refer to the integration
of the researcher to his new environment in the premises of the host. 
It does not refer to the infrastructure of the host as described in
the Quality and efficiency of the implementation criterion.}.
The application must show that the experienced
researcher will be well-integrated within the team/institution so that all parties gain
maximum knowledge and skills from the fellowship. The nature and the quality of the
research group/environment as a whole should be outlined, together with the measures
taken to integrate the researcher in the different areas of expertise, disciplines, and
international networking opportunities that the host could offer.

\medskip\noindent
For \textbf{Global Fellowships} both phases should be described - for the outgoing phase,
specify the practical arrangements in place to host a researcher coming from another
country, and for the incoming phase specify the measures planned for the successful
(re)integration of the researcher.


\subsection{Potential of the researcher to reach or re-enforce professional maturity/independence during the fellowship}
\label{sec:excellence_maturity}

Researchers should \textbf{demonstrate} how their existing professional experience, talents
and the proposed research will contribute to their development as independent/mature
researchers, \ul{\textbf{during the fellowship}}. Explain the new competences and skills that will
be acquired and how they relate to the researcher’s existing professional experience.

\medskip\noindent
Please keep in mind that the fellowships will be awarded to the most talented
researchers as shown by the proposed research and their track record (Curriculum
Vitae, section~\ref{sec:cv}), in relation to their level of experience.




\section{Impact}
\label{sec:impact}

\subsection{ Enhancing the future career prospects of the researcher after the fellowship}
\label{sec:impact_researcher}

Explain the expected impact of the planned research and training (i.e. the added value
of the fellowship) on the future career prospects of the experienced researcher \ul{\textbf{after
the fellowship}}. Focus on how the new competences and skills (as explained in \ref{sec:excellence_maturity})
can make the researcher more successful in their long-term career.


\subsection{Quality of the proposed measures to exploit and disseminate the project results}
\label{sec:impact_dissemination}

Describe how the new knowledge generated by the action will be disseminated and
exploited, and what the potential impact is expected to be. Discuss the strategy for
targeting peers (scientific, industry and other actors, professional organisations, policy
makers, etc.) and to the wider community. Also describe potential commercialisation,
if applicable, and how intellectual property rights will be dealt with, where relevant.

\medskip\noindent
For more details refer to the
\href{http://ec.europa.eu/research/participants/docs/h2020-funding-guide/grants/grant-management/dissemination-of-results_en.htm}
{"Dissemination \& exploitation" section of the H2020 Online Manual}.

\medskip\noindent
Concrete planning for exploitation and dissemination activities must be included in the
Gantt chart.


\subsection{Quality of the proposed measures to communicate the project activities to different target audiences}
\label{sec:impact_communication}

Demonstrate how the planned public engagement activities contribute to creating
awareness of the performed research. Demonstrate how both the research and results
will be made known to the public in such a way they can be understood by non-specialists.

\medskip\noindent
The type of outreach activities could range from an Internet presence, press articles
and participating in European Researchers' Night events to presenting science,
research and innovation activities to students from primary and secondary schools or
universities in order to develop their interest in research careers.

\medskip\noindent
For more details, see the guide on \href{http://ec.europa.eu/research/participants/data/ref/h2020/other/gm/h2020-guide-comm_en.pdf}
{Communicating EU research and innovation guidance for project participants}
as well as the
\href{http://ec.europa.eu/research/participants/docs/h2020-funding-guide/grants/grant-management/communication_en.htm}
{"communication" section of the H2020 Online Manual}.

\medskip\noindent
Concrete planning for communication activities must be included in the Gantt chart.




\section{Quality and Efficiency of the Implementation}
\label{sec:implementation}

\subsection{Coherence and effectiveness of the work plan, including appropriateness of the allocation of tasks and resources}
\label{sec:implementation_work_plan}

Describe how the work planning and the resources mobilised will ensure that the
research and training objectives will be reached. Explain why the number of person-
months planned and requested for the project is appropriate in relation to the proposed
activities.

\medskip\noindent
Additionally, a Gantt chart must be included in the text listing the following:

\begin{itemize}
  \item Work Packages titles (there should be at least 1 WP);
  \item Indication of major deliverables, if applicable;
  \footnote{A \textbf{deliverable} is a distinct output of the action, meaningful in terms of the action’s overall objectives and may be a report, a document, a
  technical diagram, a software, etc. Deliverable numbers should be ordered according to delivery dates. Use the numbering convention <WP
  number>.<number of deliverable within that WP>. For example, deliverable 4.2 would be the second deliverable from work package 4.}
  \item Indication of major milestones, if applicable;
  \footnote{\textbf{Milestones} are control points in the action that help to chart progress. Milestones may correspond to the completion of a key deliverable,
  allowing the next phase of the work to begin. They may also be needed at intermediary points so that, if problems have arisen, corrective
  measures can be taken. A milestone may be a critical decision point in the action where, for example, the researcher must decide which of several
  technologies to adopt for further development.}
  \item Secondments, if applicable.
\end{itemize}

\medskip\noindent
The schedule should be in terms of number of months elapsed from the start of the
action.


\begin{figure}[!htbp]
\begin{center}

\begin{minipage}{0.9\textwidth}
\vspace{2pt}
\textbf{\footnotesize 
Notes:
\begin{itemize}
  \item The titles of the WP's indicated here do not have to be stricly followed or included in the Gantt chart for your specific proposal. Adapt as needed.
  \item The number of WPs provided here is an example only. Add or remove WP's as needed.
  \item Remove any columns for a duration longer than that of your proposal.
  \item Add as much detail as needed for your proposal.
\end{itemize}
}
\end{minipage}

\begin{ganttchart}[
    canvas/.append style={fill=none, draw=black!5, line width=.75pt},
    hgrid style/.style={draw=black!5, line width=.75pt},
    vgrid={*1{draw=black!5, line width=.75pt}},
    title/.style={draw=none, fill=none},
    title label font=\bfseries\footnotesize,
    title label node/.append style={below=7pt},
    include title in canvas=false,
    bar label font=\small\color{black!70},
    bar label node/.append style={left=2cm},
    bar/.append style={draw=none, fill=black!63},
    bar progress label font=\footnotesize\color{black!70},
    group left shift=0,
    group right shift=0,
    group height=.5,
    group peaks tip position=0,
    group label node/.append style={left=.6cm},
    group progress label font=\bfseries\small
  ]{1}{24}
  \gantttitle[
    title label node/.append style={below left=7pt and -3pt}
  ]{Month:\quad1}{1}
  \gantttitlelist{2,...,24}{1} \\
  \ganttgroup{Work Package}{1}{24} \\
  \ganttgroup{Deliverable}{24}{24} \\
  \ganttgroup{Milestone}{5}{5} \\
  \ganttgroup{Secondment}{20}{23} \\
  \ganttgroup{Short stay}{16}{16} \\
  \ganttgroup{Training}{5}{5} \\
  \ganttgroup{Dissemination}{23}{24} \\
  \ganttgroup{Communication}{12}{12} \\
  \ganttgroup{Other}{18}{21}
\end{ganttchart}

\end{center}
\caption{Example Gantt Chart}
\end{figure}


\subsection{Appropriateness of the management structure and procedures, including risk management}
\label{sec:implementation_management}

Describe the organisation and management structure, as well as the progress
monitoring mechanisms put in place, to ensure that objectives are reached. Discuss the
research and/or administrative risks that might endanger reaching the action objectives
and the contingency plans to be put in place should risk occur.

\medskip\noindent
If applicable, discuss any involvement of an entity with a capital or legal link to the
beneficiary (in particular, the name of the entity, type of link with the beneficiary and
tasks to be carried out).

\medskip\noindent
If needed, please indicate here information on the support services provided by the
host institution (European offices, HR services...).


\subsection{Appropriateness of the institutional environment (infrastructure)}
\label{sec:implementation_infrastructure}

The active contribution of the beneficiary to the research and training activities should
be described. For Global Fellowships the role of partner organisations in Third
Countries for the outgoing phase should also appear.

\medskip\noindent
Give a description of the main tasks and commitments of the beneficiary and all
partner organisations (if applicable).

\medskip\noindent
Describe the infrastructure, logistics, facilities offered insofar as they are necessary for
the good implementation of the action.


\markEndPageLimit
