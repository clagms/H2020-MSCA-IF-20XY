% !TEX root = ./IF-2020-Part_B.tex

\newpage
\section*{Part B-2 Section 4 - CV of the experienced researcher (indicative length: 5 pages)}
\label{sec:cv}

The CV is intrinsic to the evaluation of the whole proposal and is assessed throughout the
three evaluation criteria by the expert evaluators. Ensure that the information provided in
Parts A and B is fully consistent. Always mention full dates (dd/mm/yyyy) in your CV.

\medskip\noindent
The CV should be limited to a maximum of 5 pages and should include \textbf{the standard
academic and research record}. Any research career gaps and/or unconventional paths
should be clearly explained so that this can be fairly assessed by the independent
evaluators. At a minimum, the CV should contain:

\begin{enumerate}[label=\alph*)]
\item the \textbf{name} of the researcher
\item \textbf{professional experience} (in \textbf{reverse} chronological order, using \textbf{exact} dates)
\item \textbf{education} (in reverse chronological order, using \textbf{exact} dates)
\end{enumerate}

\medskip\noindent
The CV should also include information on:

\begin{enumerate}
\item \textbf{Publications} in peer-reviewed scientific journals, peer-reviewed conference proceedings and/or monographs of their respective research fields, indicating also the number of citations (excluding self-citations) they have attracted.
\item Granted \textbf{patent(s)}.
\item \textbf{Research monographs, chapters} in collective volumes and any translations thereof.
\item \textbf{Invited presentations} to peer-reviewed, internationally established conferences and/or international advanced schools.
\item \textbf{Research expeditions} led by that the experienced researcher. 
\item \textbf{Organisation of International conferences} in the field of the researcher (membership in the steering and/or programme committee).
\item Examples of \textbf{participation in industrial innovation}.
\item \textbf{Prizes and Awards}.
\item \textbf{Funding} received so far.
\item \textbf{Supervising} and \textbf{mentoring} activities.
\end{enumerate}




\newpage
\section*{Part B-2 Section 5 - Capacity of the Participating Organisations}
\label{sec:capacity}

\ul{List of participating organisations (one page)}

\medskip\noindent
Please provide a list of all participating organisations (the beneficiary and, where
applicable, the entity with a capital or legal link to the beneficiary and the partner
organisation
\footnote{All partner organisations should be listed here, including secondments}
) indicating the legal entity name, the department carrying out the work and
the supervisor.

\medskip\noindent
If a secondment in Europe is planned but the partner organisation is not yet known, as a
minimum the type of organisation planned (academic/non-academic) must be stated.

\medskip\noindent
Any inter-relationship between the participating organisation(s) or individuals and other
entities/persons (e.g. family ties, shared premises or facilities, joint ownership, financial
interest, overlapping staff or directors, etc.) \textbf{must} be declared and justified \textbf{in this part of
the proposal}.

\newcommand\MyHeadLeft[2]{\multicolumn{1}{|l|}{\parbox{#1}{\centering #2}}}
\newcommand\MyHead[2]{\multicolumn{1}{l|}{\parbox{#1}{\centering #2}}}

\medskip
\medskip
\noindent\begin{tabular}{|l|m{1cm}|c|c|c|}
\hline
  \MyHeadLeft{2.4cm}{\textbf{Participating\\organisations}}
& \MyHead{1cm}{\textbf{Legal\\Entity\\Short\\Name}}
& \textbf{Country}
& \textbf{Supervisor}
& \textbf{Role of partner organisation\footnotemark} \\
\hline
\ul{Beneficiary} & & & & \\\hline
- NAME  & & & & \\\hline
\ul{Entity with a capital or legal link} & & & & \\\hline
- NAME  & & & & \\\hline
\ul{Partner Organisation} & & & & \\\hline
- NAME  & & & & \\\hline
\end{tabular}
\vspace{\baselineskip}
\footnotetext{For example hosting secondments, for GF hosting the outgoing phase, etc.}


\begin{table}[h!]
{\fontsize{9bp}{1em}\selectfont % should be 9pt
\noindent\begin{tabular}{|>{\raggedright}p{.25\textwidth}|p{.7\textwidth}|}\hline
\multicolumn{2}{|l|}{\cellcolor{gray!50}
\begin{minipage}{0.90\textwidth}
\vspace{2pt}
1 page for each role \---- chose one of:
\begin{itemize}[noitemsep,topsep=3pt]
\item beneficiary (compulsory)
\item entity with a capital or legal link to the beneficiary (optional)
\item partner organisation for GF (compulsory for GF only)
\item partner organisation for secondment (optional)
\end{itemize}
\vspace{2pt}
\end{minipage}}
\\\hline
\multicolumn{2}{|c|}{\cellcolor{gray!50}\textbf{[Full name + Legal Entity Short Name + Country]}} \\\hline
\textbf{General Description} &

\\\hline
\textbf{Academic organisation} &
(Yes / No) delete as appropriate
\\\hline
\textbf{Role and profile of key persons (supervisor)} &
{\em (names, title, qualifications of the main supervisor)}
\\\hline
\textbf{Dept./Division / Laboratory} &

\\\hline
\textbf{Key research facilities, Infrastructure and Equipment} &
{\em Demonstrate that the beneficiary has sufficient facilities and
infrastructure to host and/or offer a suitable environment for
training and transfer of knowledge to the recruited
experienced researcher.
\newline
If applicable, indicate the name of the entity with a capital or
legal link to the beneficiary and its role in the action in the
following table.}
\\\hline
\textbf{Independent research premises?} &
{\em Explain the status of the beneficiary's research facilities \---- i.e.
are they owned by the beneficiary or rented by it? Are its
research premises wholly independent from other entities?
\newline
If applicable, indicate the name of the entity with a capital or
legal link to the beneficiary and describe the nature of the
link in the following table.}
\\\hline
\textbf{Previous and current involvement in research and training programmes} &
{\em Indicate up to 5 \textbf{relevant} EU, national or international
research and training actions/projects in which the
beneficiary has previously participated and/or is currently
participating.}
\\\hline
\textbf{Relevant publications and/or research/innovation products} &
{\em (Max 5) Only list items (co-)produced by the supervisor}
\\\hline
\end{tabular}}
\end{table}




\newpage
\section{Part B-2 Section 6 - Ethical Issues}
\label{sec:ethics}

Compliance with the relevant ethics provisions is essential from the beginning to the end of
the action and is an integral part of research funded by the European Union within Horizon 2020. 

\medskip\noindent
Applicants submitting research proposals for funding with Marie Sk\l{}odowska-Curie actions in
Horizon 2020 should demonstrate proactively that they are aware of, and will comply with, European
and national legislation and fundamental ethical principles, including those reflected in the 
\href{http://www.europarl.europa.eu/charter/pdf/text_en.pdf}{Charter of Fundamental Rights of the European Union}
and the \href{http://www.echr.coe.int/Documents/Convention_ENG.pdf}{European Convention on Human Rights and its Supplementary Protocols}.

\medskip\noindent
Please be aware that it is the applicants' responsibility to identify any potential ethical issue, 
to handle the ethical aspects of the proposal and to detail how these aspects will be addressed.

\bigskip\noindent
{\large {\bf \ul{The Ethics Review Procedure in Horizon 2020}}}

\medskip\noindent
All proposals above threshold and considered for funding will be subject to an Ethics Review carried out by independent ethics experts. 
When submitting a proposal to Horizon 2020, all applicants are required to complete an Ethics Issues Table (EIT) in the Part A of the proposal. 
Applicants who flag ethical issues in the EIT have to also complete a more in-depth Ethics Self-Assessment in Part B.

\medskip\noindent
The ethics self-assessment will become part of the Grant Agreement and may thus lead to
binding obligations. The Grant Agreement can only be signed if all ethics requirement
have been duly addressed. The ethics review result will distinguish between ethics
requirements to be addressed before Grant Agreement signature and those that can be
cleared at a later stage (e.g. ethics approvals to be submitted before the start of the action
task). In the latter case, a separate work package 'Ethics Requirements' listing the
deliverables will be created automatically.

\medskip\noindent
\setlength{\fboxsep}{1mm}
\fbox{\parbox{\textwidth}{
For more details, please refer to the H2020
\href{http://ec.europa.eu/research/participants/data/ref/h2020/grants_manual/hi/ethics/h2020_hi_ethics-self-assess_en.pdf}{``How to complete your Ethics Self-Assessment''} guide.
}}

\bigskip\noindent
{\large {\bf \ul{Ethics Self-Assessment (Part B)}}}

\medskip\noindent
The Ethics Self-Assessment must:

{\bf
\begin{enumerate}[leftmargin=*, label=\arabic*)]
  \item Describe how the proposal meets the EU and national legal and ethics requirements of the country/countries where the task raising ethical is to be carried out.
\end{enumerate}
}

\medskip\noindent
For more information on how to deal with Third Countries\footnote{In the context of ethics
appraisal, Third Country refers to non-EU country; Associated Countries are "ethics"
TC} please see Article 34 of the
\href{http://ec.europa.eu/research/participants/data/ref/h2020/grants_manual/amga/h2020-amga_en.pdf}{Annotated Model Grant Agreement},
as well as the
\href{http://ec.europa.eu/justice/data-protection/international-transfers/adequacy/index_en.htm}{rules for the protection of personal data inside and outside the EU}.
Please ensure and confirm that the research performed outside the EU
is compatible with the Union, National and International legislation and could have
been legally conducted in one of the EU Member States.

\medskip\noindent
Please list the documents provided with their expiry date.

\medskip\noindent
Ensure early compliance of the proposed research with EU and national legislation on
ethics in research. Should your proposal be selected for funding, you will be required - if
applicable - to confirm that you have obtained the following documents needed for
implementing the action tasks in question:

\begin{enumerate}[label=(\alph*)]
  \item any ethics committee opinion required under national law and
  \item any notification or authorisation for activities raising ethical issues required under national and/or European law
\end{enumerate}

\medskip\noindent
under national and/or European law
If you have not already applied for/received the ethics approval/required ethics
documents when submitting the proposal, please indicate in this section the approximate
date when you will obtain the relevant approvals/authorisations and any other ethics
documents. Please state explicitly that you will not proceed with any research with ethical
implications before obtaining the necessary authorizations/opinions.

\medskip\noindent
\fbox{\parbox{\textwidth}
{\em The documents must be kept on file and be submitted upon request by the beneficiary to
the REA (see Article 52). If they are not in English, they must be submitted together with
an English summary, which shows that the action tasks in question are covered and
includes the conclusions of the committee or authority concerned (if available).

\medskip\noindent
If you plan to request these ethics documents specifically for your proposed action, your
request must contain an explicit reference to the action's title.
}}

\bigskip
{\bf 
\begin{enumerate}[leftmargin=*, label=\arabic*), start=2]
  \item Explain in detail how you intend to address the ethical issues flagged, in particular with regard to: 
  \begin{itemize}
    \item {\normalfont the research {\bf objectives} (e.g., study of vulnerable 
    populations, cooperation with a Third Country, etc.);}
    \item {\normalfont the research {\bf methodology} (e.g., clinical trials, 
    involvement of children and related information and consent/assent procedures, data protection and privacy issues related to data collected, etc.);}
    \item {\normalfont the potential {\bf impact} of the research (e.g. dual use issues,
    environmental damage, malevolent use, etc.);}
    \item {\normalfont appropriate health and safety procedures - conforming to relevant
    local/national guidelines/legislation - for the staff involved;}
    \item {\normalfont  possible harm to the environment the research might cause
    (e.g. environmental risks of nanomaterials), and measures that will be taken to mitigate the risks.}
  \end{itemize}
\end{enumerate}
}

\noindent
In order to facilitate the ethics review of the proposal, you may wish to include in this section
one of the following statements (if relevant/applicable). Please delete as appropriate:

\begin{table}[h!]
{\fontsize{9bp}{1em}\selectfont % minimum size of 8pt
\noindent\begin{tabular}{|>{\raggedright}p{.85\textwidth}|cc|}\hline
\multicolumn{3}{|l|}{\cellcolor{gray!50} \textbf{Humans}} \\\hline
I confirm that templates of the informed consent forms and information
sheets (in language and terms intelligible to the participants) will be kept on file.
& Yes & No \\\hline
\multicolumn{3}{|l|}{\cellcolor{gray!50} \textbf{Animals}} \\\hline
I confirm that training certificates/personal licenses of the staff involved in
animal experiments have been obtained and will be kept on file.
& Yes & No \\\hline
I confirm that relevant authorisations for animal experiments (covering also
the work with genetically modified animals, if applicable) have been
obtained, and will be kept on file.
& Yes & No \\\hline
\multicolumn{3}{|l|}{\cellcolor{gray!50} \textbf{Environmental protection and safety}} \\\hline
I confirm that appropriate health and safety procedures conforming to
relevant local/national guidelines/legislation are followed for staff involved
in this project.
& Yes & No \\\hline
I confirm that authorisations for relevant facilities (e.g. security
classification of laboratory, GMO authorisation) have been obtained, and
will be kept on file.
& Yes & No \\\hline
\multicolumn{3}{|l|}{\cellcolor{gray!50} \textbf{Third country}} \\\hline
I confirm that the research performed outside the EU is compatible with
the Union, National and International legislation and could have been
legally conducted in one of the EU Member States.
& Yes & No \\\hline
\multicolumn{3}{|l|}{\cellcolor{gray!50} \textbf{Data protection}} \\\hline
I confirm that a Data Protection Officer (DPO) has been appointed and the
contact details of the DPO are made available to all data subjects involved
in the research.
& Yes & No \\\hline
I confirm that data intended to be processed is relevant and limited to the
purposes of the research project (in accordance with the 'data minimisation'
principle).
& Yes & No \\\hline
I confirm that relevant authorisations for further processing of previously
collected personal data have been obtained and will be kept on file.
& Yes & No \\\hline
I confirm that the data used are publicly available.
& Yes & No \\\hline
\end{tabular}}
\end{table}




\newpage
\section{Part B-2 Section 7 - Letter of commitment (Global Fellowship only)}
\label{sec:letters}

For Global Fellowship proposals, a {\em letter of commitment} \textbf{of the partner
organisations} (hosting the outgoing phase in a Third country) must be included in Part
B-2 to ensure their real and active participation. 
Do not attach this letter as a separate PDF file or as an
embedded file since this makes them invisible in the proposal. GF Proposals which fail to
include a letter of commitment of the partner organisation will be declared \textbf{inadmissible}.

\medskip\noindent
Minimum requirements for the letter of commitment:

\begin{itemize}
  \item heading or stamp from the institution;
  \item up-to-date (may not be dated prior to the call publication);
  \item the text must demonstrate the will to actively participate in the (identified) proposed action and the precise role.
\end{itemize}

\noindent
Please note that no template for this letter is provided, only general indications.

